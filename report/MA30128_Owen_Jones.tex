\documentclass[10pt]{article}

\usepackage{amssymb, amsmath}
\usepackage[a4paper]{geometry}
\geometry{left={2cm}, right={2cm}, top={2cm}, bottom={2cm}}
%\geometry{left={3cm}, right={3cm}, top={3cm}, bottom={3cm}}

\begin{document}
\thispagestyle{empty}
%% Config
\newcommand{\Title}{}
\newcommand{\Student}{Owen Jones}
\newcommand{\Supervisor}{Dr James Hook}   %That's you!
\newcommand{\Checker}{Dr Melina Freitag}
\newcommand{\HoG}{Prof Roger Moser}
%% Unit code.  Choose from:
%% MMath S1=MA40117
%% BSc S2=MA30128
%% MMath S2=MA40195
\newcommand{\Unit}{MA30128} 
\newcommand{\Deadline}{8th May 2019}

\setlength{\topsep}{0pt}
\setlength{\partopsep}{0pt}
\setlength{\parindent}{0pt}
\setlength{\parskip}{6pt}

\begin{center}
\bfseries{\Large \Unit\ Project}\\[6pt]
\large\Title Neural Networks for Classification \end{center}
\medskip
\begin{center}
\begin{tabular}{ll}
Student:&\Student\\
Supervisors:&\Supervisor\\
Checker/Second examiner:&\Checker\\
Head of Group:&\HoG\\
Final submission deadline:&\Deadline
\end{tabular}
\end{center}
\vspace{-0.4cm}
\section*{Description}
%% Blurb that describes the project

Machine learning in general and deep learning in particular have revolutionized our modern understanding of how data can be used and what information can be extracted from it, see \cite{Higham2} and references therein. In this project the student will learn about Neural Networks and how they can be used to solve classification and regression problems. 

The (preliminary) work plan is as follows:
\begin{itemize}
\item Read and make notes from \cite{Higham2} to understand neural networks (2-3 weeks).
\item Write MATLAB or Python code from scratch to implement neural networks with variable architecture and loss functions  (2 weeks).
\item Apply above code to at least one contest problem from Kaggle (www.kaggle.com) (2 weeks).
\item As time allows, explore Bayesian Optimization as a method for hyper-parameter tuning \cite{bp,ar} (2-3 weeks).
\end{itemize}
The student will give a 20 minute oral presentation (15 minutes plus 5 for questions) on his work. 

A well-structured report, explaining the methods and findings in a coherent way should be produced. It should be no longer than $15$ pages of A4 (not including codes, diagrams and figures), single spaced and typeset in a normal font size (11pt).
\vspace{-0.6cm}
\begin{thebibliography}{99}
\bibitem{Higham2}
{\sc R. M. Higham and D. J. Higham. Deep Learning: An Introduction for Applied Mathematicians, https: //arxiv.org/abs/1801.05894, 2018.}
\bibitem{bp}
{\sc  B. Shahriari \emph{et al}. Taking the Human Out of the Loop:
A Review of Bayesian Optimization,  \\ https://www.cs.ox.ac.uk/people/nando.defreitas/publications/BayesOptLoop.pdf, 2016.} 
\bibitem{ar}
{\sc K. Kandasamy \emph{et al}. Neural Architecture Search with Bayesian Optimization and Optimal Transport, \\ https://arxiv.org/abs/1802.07191, 2018.}

\end{thebibliography}
\vspace{0.2cm}

\section*{Assessment}
\begin{center}
%%%%%%%%%%%%%%%%%%%%%%%%%%%%%%%%
%% This is typical for a generic maths project
%%%%%%%%%%%%%%%%%%%%%%%%%%%%%%%%
\begin{tabular}{lr}
Personal Initiative: &10\\
Use of the Literature: &10\\
Style of Report: &10\\
Contents of Report (depth, breadth, accuracy): &50\\
Quality of Oral Presentation: &20\\ %This is now unalterable

\cline{2-2}
TOTAL&100
\end{tabular}
%%%%%%%%%%%%%%%%%%%%%%%%%%%%%
%% Suitable for Stats/Modelling
%%%%%%%%%%%%%%%%%%%%%%%%%%%%%
% \begin{tabular}{lr}
% Background/statistical theory: &25\\
% Computing : &20\\
% Analysis and write-up: &35\\
% Oral presentation: &20\\%Unalterable
% \cline{2-2}
% TOTAL&100
% \end{tabular}
% \end{center}
%%%%%%%%%%%%%%%%%%%%%%%%%%%%%%%
%% Suitable for Math. Biol.
%%%%%%%%%%%%%%%%%%%%%%%%%%%%%%%
% \begin{tabular}{lr}
% Personal Initiative: &10\\
% Presentation of Report: &10\\
% Introduction (mathematical background, biology): &10\\
% Modelling: &20\\
% Analysis of models: &20\\
% Conclusions and interpretation: &10\\
% Oral presentation: &20\\ %Unalterable
% \cline{2-2}
% TOTAL&100
% \end{tabular}
\end{center}
\section*{Signatures}

\begin{tabular}[t]{*{4}{p{1.5in}}}
Student&Supervisor&Checker&Head of Group\\[1in]
\Student&\Supervisor&\Checker&\HoG
\end{tabular}

%\section*{Signatures}
%
%\begin{tabular}[t]{*{4}{p{1.5in}}}
%Student&Supervisor&Checker&Head of Group\\[1in]
%\Student&\Supervisor&\Checker&\HoG
%\end{tabular}
\end{document}
